
%%%%%%%%%%%%%%%%%%%%%%%%%%%%%%%%%%%%%%%%%%%%%%%%%%%%%%%%%%%%%%%%%%%%%%%%%%%%%
\Chapter{Methods and functions}

This chapter describes all the main methods and functions of this
package. 

%%%%%%%%%%%%%%%%%%%%%%%%%%%%%%%%%%%%%%%%%%%%%%%%%%%%%%%%%%%%%%%%%%%%%%%%%%%%
\Section{Functions for finite $p$-groups}

Let $G$ be a finite $p$-group given by a consistent polycyclic 
presentation as Pc group.

\> LCSFactorTypes( G )

returns the abelian invariants of the lower central series factors of $G$.

\> LCSFactorSizes( G )

returns the orders of the lower central series factors of $G$.

\> WidthPGroup( G )

returns the width of $G$.

\> SubgroupRank( G )

returns the (subgroup-)rank of $G$.

\> Obliquity( G )

returns the obliquity of $G$.

\> HasObliquityZero( G )

checks whether $G$ has obliquity 0 and returns true or false.

%%%%%%%%%%%%%%%%%%%%%%%%%%%%%%%%%%%%%%%%%%%%%%%%%%%%%%%%%%%%%%%%%%%%%%%%%%%%%
\Section{Functions to generate groups and trees}

Let $G(p,rwo)$ denote the full tree of all finite $p$-groups with
rank $rwo[1]$, width $rwo[2]$ and obliquity $rwo[3]$. This tree can be
finite or infinite; if it is infinite, then the infinite
pro-$p$-groups of the considered rank, width and obliquity specify
infinite subtrees of the full tree. The groups not contained in such
an infinite subtree are called sporadic.

\> GroupsByRankWidthObliquity( p, d, rwo, roots, limit )

determines all $p$-groups $G$ with $G/\Phi(G)$ of order $p^d$ and
rank, width and obliquity as prescribed in $rwo$ up to order $limit$.
Here $p$ and $d$ are integers, $rwo$ is a list of three integers and
$limit$ is an integer. 

The parameter $roots$ is a list of groups described by their id's 
with respect to the small groups library. The descendants of the 
groups described in $roots$ are excluded from the output of this 
function. This option can be used to prune the tree of groups 
determined by this function.

If there are only finitely many sporadic $p$-groups with given rank,
width and obliquity, then this function can be used to generate them;
in this case $roots$ must contain a complete list of all id's of roots
of infinite subtrees and $limit$ can be set to infinity.

\> BranchRWO( G, i, rwo )

for a stable  quotient (see \cite{ER10}) $G$ of a pro-$p$-group of
rank $rwo[1]$, $rwo[2]$ and obliquity $rwo[3]$, this function returns
the $i$-th branch of its corresponding tree. The structure of the tree
is encoded in a list. If one of the global parameters `CHECK_RANK' or 
`CHECK_OBLIQUITY' is set to false, then checking the corresponding
invariant is omitted and hence a potentially larger tree is returned. 

The user is advised not to perform any other computations using ANUPQ
or the `pq'-program while using this or the following function,
because such computations will be terminated.

\> BoundedDescendantsRWO( G, i, c, rwo)

returns the tree of all descendants of $G/\gamma_i(G)$ of rank $rwo[1]$,
width $rwo[2]$, obliquity $rwo[3]$ and class at most $c$.

\> DrawBranch( branch )

if the package is run under XGap, then this function can be used to draw
a branch as output by the above two functions in the case of width 2.
The user may wish to improve the quality of the output by modifying
the file `gap/xbranch.gi'.

Vertices drawn on the same level correspond to groups of the same
class. If $G$ is a descendant of $H$ in the branch, then $G$ is drawn
as a filled circle if $\vert G\vert = \vert H\vert p$ and as a solid
box if $\vert G\vert = \vert H\vert p^2$.

%%%%%%%%%%%%%%%%%%%%%%%%%%%%%%%%%%%%%%%%%%%%%%%%%%%%%%%%%%%%%%%%%%%%%%%%%%%%%

%\Section{Quotients of pro-$p$-groups}

The package also provides finite quotients of a number of infinite
pro-$p$-groups with finite rank, width and obliquity. Throughout the
section, $p$ is an odd prime.

\> ProPSylowGroupOfPSL(d, p, n)

returns the quotient of the Sylow pro-$p$-subgroup of
${\rm PSL}_d({\Q}_p)$ modulo the matrices which are congruent to the
identity modulo $p^n$. 
  
\> ProPSylowGroupOfPSF(p, n)

Let $L$ be the simple Lie algebra of dimension $3$ over $\Q_p$
which is not isomorphic to $sl_2(\Q_p)$. This function  returns a
finite quotient of the Sylow pro-$p$-subgroup of its automorphism  
group. The parameter $n$ specifies how large this quotient is.

In \cite{KLGP97}, a library of maximal pro-$p$-groups with finite
rank, width and obliquity corresponding to the Lie algebras of small
dimension is provided. Here, we provide a library of large quotients
of these groups for some of the Lie algebras of type $sl_d(K)$, where
$K$ is a finite extension of $\Q_p$. 
These groups have been determined using the programs described in
\cite{KLGP97}. To be precise, depending on the group, it may be
necessary to pass to the quotient by one of the last non-trivial terms
of the lower central series in order to obtain a quotient of the
respective pro-$p$-group.

\> ProPQuotient(p, dim, deg, no)

returns a finite group corresponding to the maximal pro-$p$-group $G$
with Lie algebra $sl_{dim}(K)$, where $K$ is a field of
degree $deg$ over $\Q_p$. The parameter $no$ specifies the number of
the group in our database.

\Section{Example}

When run under XGap, the following code constructs and draws the
branch with root $G/\gamma_5(G)$ in the graph of finite 5-groups of
rank 3, width 2 and obliquity 0, where $G$ is the Sylow
pro-$p$-subgroup of ${\rm Aut}(sl_2(\Q_5))$.

\beginexample
gap> g := ProPSylowGroupOfPSL(2,5,6);
Pcp-group with orders [ 5, 5, 5, 5, 5, 5, 5, 5, 5, 5, 5, 5, 5, 5, 5, 5 ]
gap> branch := BranchRWO(g,5,[3,2,0]);;
ConstructBranch: root-p-class: 4
Constructed 3 1-step descendants.
ConstructBranch: root-p-class: 5
Constructed 0 1-step descendants.
Constructed 0 2-step descendants.
ConstructBranch: root-p-class: 5
Constructed 0 1-step descendants.
Constructed 0 2-step descendants.
ConstructBranch: root-p-class: 5
Constructed 0 1-step descendants.
Constructed 0 2-step descendants.
Constructed 3 2-step descendants.
ConstructBranch: root-p-class: 5
Constructed 0 1-step descendants.
Constructed 0 2-step descendants.
ConstructBranch: root-p-class: 5
Constructed 0 1-step descendants.
Constructed 0 2-step descendants.
gap> DrawBranch(branch);
\endexample

A window with the following graph should appear.

\epsfxsize=8cm \epsfbox{branch.ps}